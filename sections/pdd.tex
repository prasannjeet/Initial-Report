For the sake of consistency in this paper, we will follow the taxonomy for dependable and secure computing as proposed by Avizienis et al. [33]. A state of the system or a portion of state of the system that may entail its eventual failure is labelled as an error. Further, the speculated cause of the error is termed as a fault; and finally, when the functionality of the system diverges from its correct behavior, it is termed as a failure. A fault, when activated causes an error, and the propagation of this error causes a failure [41]. 

Since the expected number of bugs in a system is proportional to its lines of code [43], it can be safely assumed that it is impossible to exhaustively list down all possible faults in a system. Therefore, introduction of faults in a system is inevitable, and more emphasis is given in developing a fault-tolerant system, rather than building a fault-less system. A system could be termed as fault-tolerant, if it is able to prevent the fault from turning into a failure. [45] proposes a fault tolerant mechanism which is implemented on a WuKong[46] middleware. The mechanism furnishes failover for application components to comply with the requirements of fault tolerance. To prevent bottleneck throttling, it capitalizes the decentralized nature of IoT devices by adopting a decentralized algorithm. [47] proposes a fault-tolerant architecture for healthcare IoT systems consisting Wireless Sensor Networks (WSN). A 6LoWPAN (IPv6 over Low-Power Wireless Personal Area Networks) architecture is presented for healthcare environments that constructs an improved gateway that solves the bottleneck problem in edge routers due to the limitations in 6LoWPAN, thus improving the network fault tolerance. Other faults such as the malfunction of the sink node[48] hardware are also covered. Mishra et al. in their research [49] propose a method for fault-tolerant routing in IoT. A fault-tolerant routing protocol is suggested for IoT systems that is based on mixed cross-layer and learning automata (LA). A successful packet delivery between nodes is assured even with the existence of faults. Kouwe, in their thesis[50] propose fault-injection, where artificial faults are injected into a system in order to learn the behavior of a system with activated faults, and eventually prevent system failure.

Cyber Physical Systems such as SCADA (Supervisory Control And Data Acquisition) is an appropriate example for the current state-of-the-art in fault tolerant Internet of Things system [51]. These systems are competent in offering flexibility, stability and fault tolerance. These systems exploiting cloud computing services, integrated with Internet of Things can be judged as Smart Industrial Systems which are predominantly employed in smart grids, smart transportation, eHealthcare and smart medical systems.

Intentional attacks are common threats in IoT. Thus, it is important to employ an IoT intrusion prevention mechanism in to prevent those threats. Various methods of intrusion detection are discussed and a taxonomy for the same is proposed by Zarpelãoa et al. [54]. Bertino and Islam [55] propose various guidelines that can prevent an IoT system from being compromised. Kasinathan et al. [56] propose a DoS Detection Architecture for 6LoWPAN IoT systems using Suricata, an open-source intrusion detection system that detects and eliminates the attack using appropriate countermeasures before the network operations are disrupted. Another efficient way of intrusion prevent is the installation of Honeypot in a system. Likewise, Honeynets are an aggregation of Honeypots that are intended to imitate usual servers and network services [57]. In general, honeypots are essentially a technique of deception, where the defender purposely hoodwinks the attacker into acting in one’s favor [58]. While Yu et al. [59] rejects the idea of Honeypots in an IoT setup due to non-scalability and dependency issues, La et al. [60] considers the possibility and uses game theory to analyze the situation where both attacker and defender try to deceive each other. 
While virtual patching is a type of firewall often mentioned as a Web Application Firewall (WAF), an IoT system also needs a full-fledged firewall as most of the embedded systems that are part of the IoT system have little to no security. A recent improvement on the traditional firewalls for IoT is Smart Firewall, which, unlike traditional software-based firewall, is a hardware-based device [69]. Gupta et al. [70] have proposed an implementation of firewall for Internet of Things using Raspberry Pi as  a gateway. Other legacy research works in the field of firewall and security of IoT include [71][72][73][74].
