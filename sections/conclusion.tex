As we see the importance and growing presence of IoT and Connected Cyber-Physical systems around the world, we stressed the fact that there is need for methodologies, tools, technologies, procedures and frameworks supporting automatic error detection and – possibly – repair.  However, bisecting process models requires various protocols to be followed\cite{Aalst2015}, which are not covered in this survey. Once a robust process model is built, the following things can easily be found:

\begin{enumerate}
	\item Something that should have happened but did not.
	\item Something that shouldn’t have happened, happened.
	\item If the IoT system is being used as it was intended to be.
	\item Models can also be used to analyze what is the most frequent path followed by the user, etc.
\end{enumerate}

There are various tools available in the market to create a process model, e.g. ProM, Disco, etc.. However, to create a process model for a new system, a plugin may be needed so that the event logs can easily by comprehended by the tool. Note that apart from self-healing, process mining may also capable of solving the problem of error predictability, where an end-user may be warned beforehand, if their actions have a high probability of causing an error.
